\documentclass[a4paper]{article}
\usepackage[utf8]{inputenc}
\usepackage[czech]{babel}
\usepackage{xcolor}
\usepackage{fancyhdr}

\usepackage[left=2cm,text={17cm, 24cm}, top=3cm]{geometry}
\usepackage{amsmath} 
\usepackage{enumitem}
\usepackage{multirow}
\usepackage[european,straightvoltages,betterproportions,EFvoltages]{circuitikz}
\usepackage{siunitx}
\sisetup{output-decimal-marker = {,}, group-minimum-digits=4}
\def\doubleunderline#1{\underline{\underline{#1}}}
\makeatletter
\newcommand*{\rom}[1]{\expandafter\@slowromancap\romannumeral #1@}
\makeatother

\begin{document}

    %Titulni strana
    \begin{titlepage}
        \begin{center}
            \includegraphics[width=0.87\textwidth]{images/logo_cz.png}
            \vspace*{6cm}

            \Large
            \textbf{ELEKTRONIKA PRO INFORMAČNÍ TECHNOLOGIE}    
            
            \vspace{0.5cm}
            \Large
            \textbf{2020/2021}
            
            \vspace{2cm}
            
            \Large
            \textbf{Semestrální projekt}
            
           \vfill
		   \begin{flushleft} 
		   \large
		   David Chocholatý (xchoch09)
		   \hfill
		   Brno, \today
		   \end{flushleft}
            
        \end{center}
    \end{titlepage}

\pagestyle{fancy}
\lhead{\bfseries IEL projekt 2020/2021}
\rhead{\bfseries Obsah}

\newpage
\tableofcontents
\newpage

%Priklad 1
\rhead{\bfseries Příklad 1}
\section{Příklad 1 -- Metoda postupného zjednodušování obvodu}
\textbf{Zadání:} Stanovte napětí $U_{R_6}$ a proud $I_{R_6}$. Použijte metodu postupného zjednodušování obvodu.

\begin{table}[ht]
  \begin{center}
    \begin{tabular}{|c|c|c|c|c|c|c|c|c|c|c|} 
      \hline
      %prvni radek
       sk. & $U_1$ [\si{\volt}] & $U_2$ [\si{\volt}] & $R_1$ [\si{\ohm}] & $R_2$ [\si{\ohm}]
       & $R_3$ [\si{\ohm}] & $R_4$ [\si{\ohm}] & $R_5$ [\si{\ohm}]
       & $R_6$ [\si{\ohm}] & $R_7$ [\si{\ohm}] & $R_8$ [\si{\ohm}]\\
       %druhy radek
       \hline
       H & 135 & 80 & 680 & 600 & 260 & 310 & 575 & 870 & 355 & 265 \\
     \hline
    \end{tabular}
    \caption{Zadané hodnoty}
    \label{tab:1}
  \end{center}
 \end{table}

%obvod 1
\begin{figure}[ht!]
\begin{center}
\begin{circuitikz}
    \draw
    (0,0) to[dcvsource, v^<=$U_2$] ++(0,2)%zdroj U2
    to[dcvsource, v^<=$U_1$] ++(0,2)%zdroj U1
    to[short, -*] (1,4)
    to[R=$R_1$, -*] (4,4) -- (7,4)
    to[R=$R_7$, -*] (7,2) -- (7,0)
    (7,0) -- (5,0)
    to[R=$R_8$] (0,0);
    
    \draw
    (1,4) to[R=$R_2$, -*] (1,2)
    to[R=$R_3$] (4,2)
    to [R=$R_6$, v>=$U_{R_6}$, i>^=$I_{R_6}$] (7,2);
    
    \draw
    (1,2)--(1,1) to[R=$R_4$] (4,1)
    to[short, -*] (4,2);
    
    \draw
    (4,4) to[R=$R_5$, -*] (4,2);
     
\end{circuitikz}
\caption{Zadaný obvod}
\end{center}
\end{figure}

\subsection{Zjednodušení obvodu}
Jako první krok při zjednodušování zadaného obvodu zvolíme nahrazení zdrojů napětí $U_1$ a $U_2$ 
zapojených \newline v sérii zdrojem napětí $U_V$ dle \rom{2}. Kirchhoffova zákona:
\[U_1 + U_2 - U_V = 0\] 
po úpravě: 
\[U_V = U_1 + U_2 = 135 + 80 = 215V\]

%obvod 2
\begin{figure}[ht!]
\begin{center}
\begin{circuitikz}
    \draw
    (0,0) to[dcvsource, v^<=$U_v$] ++(0,4)
    to[short, -*] (1,4)
    to[R=$R_1$, -*] (4,4) -- (7,4)
    to[R=$R_7$, -*] (7,2) -- (7,0)
    (7,0) -- (5,0)
    to[R=$R_8$] (0,0);
    
    \draw
    (1,4) to[R=$R_2$, -*] (1,2)
    to[R=$R_3$] (4,2)
    to [R=$R_6$, v>=$U_{R_6}$, i>^=$I_{R_6}$] (7,2);
    
    \draw
    (1,2)--(1,1) to[R=$R_4$] (4,1)
    to[short, -*] (4,2);
    
    \draw
    (4,4) to[R=$R_5$, -*] (4,2);
    
\end{circuitikz}
\caption{Nahrazení zdrojů napětí $U_1$ a $U_2$ zdrojem napětí $U_V$}
\end{center}
\end{figure}
\newpage

\noindent
Dalším krokem ke zjednodušení obvodu je nahradit rezistory $R_3$ a $R_4$ zapojené paralerně rezistorem $R_{34}$ podle vztrahu:

\[R_{34} = \frac{R_3 * R_4}{R_3 + R_4} = \frac{260 * 310}{260 + 310} = \SI{141,403 509}{\ohm}\]

%obvod 3
\begin{figure}[ht!]
\begin{center}
\begin{circuitikz}
    \draw
    (0,0) to[dcvsource, v^<=$U_v$] ++(0,4)
    to[short, -*] (1,4)
    to[R=$R_1$, -*] (4,4)
    to[R=$R_7$] (7,4)--(7,0)
    (7,0) -- (5,0)
    to[R=$R_8$] (0,0);
    
    \draw
    (1,4) to[R=$R_2$] (1,2)
    to[R=$R_{34}$, -*] (4,2)
    to [R=$R_6$, -*] (7,2);
     
     \draw
     (4,4) to[R=$R_5$] (4,2);
     
\end{circuitikz}
\caption{Nahrazení rezistorů $R_3$ a $R_4$ rezistorem $R_{34}$}
\end{center}
\end{figure}

\noindent
Nyní si můžeme povšimnout, že nově vzniklý rezistor $R_{34}$ je zapojený v sérii s rezistorem $R_2$. 
Nahradíme jej tedy rezistorem $R_{234}$ podle vztahu:
\[R_{234} = R_2 + R_{34} = 600 + \num{141,403 509} = \SI{741,403 509}{\ohm}\]

%obvod 4
\begin{figure}[ht!]
\begin{center}
\begin{circuitikz}
    \draw
    (0,0) to[dcvsource, v^<=$U_v$] ++(0,4)
    to[short, -*] (1,4)
    to[short] (1,5)
    to[R=$R_1$, -*] (4,5)
    to[R=$R_7$] (7,5)--(7,4);
    
    \draw
    (1,4) -- (1,3)
    to[R=$R_{234}$, -*] (4,3)
    to [R=$R_6$] (7,3)
    to[short, -*] (7,4) -- (8,4)
    (8,4) -- (8,0)
    to[R=$R_8$] (0,0);
     
     \draw
     (4,5) to[R=$R_5$] (4,3);
     
\end{circuitikz}
\caption{Nahrazení rezistorů $R_2$ a $R_{34}$ rezistorem $R_{234}$}
\end{center}
\end{figure}

\newpage

Při zjednodušování obvodu dále využijeme transformaci trojúhelník -- hvězda, 
která danou část obvodu, kterou chceme transformovat, zjednoduší.

V námi dosud zjednodušeném obvodu lze aplikovat transformaci trojuhelník -- hvězda dvěma způsoby, 
a to pro levou nebo pravou smyčku. Jelikož budeme počítat hodnoty proudu a napětí pro odpor $R_6$, 
který se nachází v pravé smyčce, aplikujeme transformaci na smyčku levou. 
Tato volba se nám později projeví tím, že bude jednodušší zpětně dopočítat výsledné hodnoty $U_{R_6}$ a $I_{R_6}$. 

%obvod 5
\begin{figure}[ht!]
\begin{center}
\begin{circuitikz}
    \draw
    (0,0) to[dcvsource, v^<=$U_v$] ++(0,4)
    to[short, -*] (1,4)
    node[label={left:\textcolor{red}{A}}, yshift=0.25cm] {}
    to[short] (1,5)
    to[R=$R_1$, -*] (4,5)
    node[label={above:\textcolor{red}{B}}] {}
    to[R=$R_7$] (7,5)--(7,4);
    
    \draw
    (1,4) -- (1,3)
    to[R=$R_{234}$, -*] (4,3)
    node[label={below:\textcolor{red}{C}}] {}
    to [R=$R_6$] (7,3)
    to[short, -*] (7,4) -- (8,4)
    (8,4) -- (8,0)
    to[R=$R_8$] (0,0);
     
     \draw
     (4,5) to[R=$R_5$] (4,3);
     
\end{circuitikz}
\caption{Obvod s vyznačenými uzly}
\end{center}
\end{figure}

\noindent
V zadaném obvodu aplikujeme transformaci trojúhelník -- hvězda pro rezistory $R_1$, $R_{234}$ a $R_5$.

%obrazek 6 - transformace trojuhelnik-hvezda
\begin{figure}[ht!]
\begin{center}
\begin{circuitikz}
    \draw
    (0,0) to[R=$R_{234}$, -o] (2,2)
    node[label={above:\textcolor{red}{A}}] {}
    to[R=$R_1$, -o] (4,0)
    node[label={above:\textcolor{red}{B}}] {}
    to[R=$R_5$, -o] (0,0)
    node[label={above:\textcolor{red}{C}}] {};
    
    \draw
    [-{Latex[length=5mm, width=2mm]}] (5.5,1)--(7.5,1);
    
    \draw
    (11,4) to[short, -o] (11,4)
    node[label={above:\textcolor{red}{A}}] {}
    to [R=$R_A$, -*] (11,2)
    to [R,l_=$R_C$] (9.5,0)
    to [short, -o] (9.5,0)
    node[label={above:\textcolor{red}{C}}] {};
    
    \draw
    (11,2) to [R=$R_B$] (12.5,0) 
    to [short, -o] (12.5,0)
    node[label={above:\textcolor{red}{B}}] {};
\end{circuitikz}
\caption{Transformace trojúhelník -- hvězda}
\end{center}
\end{figure}

\newpage

\noindent
Nyní můžeme trojúhelník překreslit přímo do zadaného obvodu:

%obvod 7
\begin{figure}[ht!]
\begin{center}
\begin{circuitikz}
    \draw
    (0,0) to[dcvsource, v^<=$U_v$] ++(0,4)
    to[short, -*] (1,4)
    node[label={left:\textcolor{red}{A}}, yshift=0.25cm] {}
    to[R=$R_1$, -*] (3,5)
    node[label={above:\textcolor{red}{B}}] {}
    to[R=$R_7$] (6,5)--(6,4);
    
    \draw
    (1,4) to[R=$R_{234}$, -*] (3,3)
    node[label={below:\textcolor{red}{C}}] {}
    to [R=$R_6$] (6,3)
    to[short, -*] (6,4) -- (7,4)
    (7,4) -- (7,0)
    to[R=$R_8$] (0,0);
     
     \draw
     (3,5) to[R=$R_5$] (3,3);
     
\end{circuitikz}
\caption{Zadaný obvod před transformací}
\end{center}
\end{figure}

\noindent
Aplikace transformace trojuhelník -- hvězda:

%obvod 8
\begin{figure}[ht!]
\begin{center}
\begin{circuitikz}
    \draw
    (0,0) to[dcvsource, v^<=$U_v$] ++(0,4)
    to[R=$R_A$, -*] (2,4)
    to[R=$R_B$] (4,5)
    to[R=$R_7$] (7,5)--(7,4);
    
    \draw
    (2,4) to[R=$R_C$] (4,3)
    to [R=$R_6$] (7,3)
    to[short, -*] (7,4) -- (8,4)
    (8,4) -- (8,0)
    to[R=$R_8$] (0,0);
     
\end{circuitikz}
\caption{Zadaný obvod po transformaci}
\end{center}
\end{figure}

\newpage

\noindent
Následně si obvod přeuspořádáme pro lepší názornost:

%obvod 9
\begin{figure}[ht!]
\begin{center}
\begin{circuitikz}
    \draw
    (0,0) to[dcvsource, v^<=$U_v$] ++(0,4)
    to[R=$R_A$, -*] (2,4)
    to[short] (2,5)
    to[R=$R_B$] (4.5,5)
    to[R=$R_7$] (7,5)--(7,4);
    
    \draw
    (2,4) to[short] (2,3)
    to[R=$R_C$] (4.5,3)
    to [R=$R_6$] (7,3)
    to[short, -*] (7,4) -- (8,4)
    (8,4) -- (8,0)
    to[R=$R_8$] (1,0) -- (0,0);
     
\end{circuitikz}
\caption{Obvod po přeuspořádání}
\end{center}
\end{figure}

\noindent
Při odvozován vztahů pro převod trojúhelníku na hvězdu je možný zápis soustavy rovnic ve tvaru:
\[R_A + R_B = \frac{R_1 * (R_{234} + R_5)}{R_1 + R_{234} + R_5}\]
\[R_A + R_C=\frac{R_{234} * (R_1 + R_5)}{R_1 + R_{234} + R_5}\]
\[R_B + R_C=\frac{R_5 * (R_1 + R_{234})}{R_1 + R_{234} + R_5}\]

\noindent
Po úpravě rovnic dostáváme 3 vztahy pro $R_A$, $R_B$ a $R_C$:
\[R_A = \frac{R_1 * R_{234}}{R_1 + R_{234} + R_5} = \frac{680 * \num{741,403 509} }{680 + \num{741,403 509} + 575} = \SI{252,531 306}{\ohm}\]

\[R_B = \frac{R_1 * R_5}{R_1 + R_{234} + R_5} = \frac{680 * 575}{680 + \num{741,403 509} + 575} = \SI{195,852 19}{\ohm}\]

\[R_C = \frac{R_{234} * R_5}{R_1 + R_{234} + R_5} = \frac{\num{741,403 509} * 575}{680 + \num{741,403 509} + 575} = \SI{213,537 502}{\ohm}\]

\newpage

\noindent
V takto transformovaném obvodu budeme dále zjednodušovat podle sériového či paralerního zapojení rezistorů. 
Nyní můžeme nahradit rezistory $R_B$ a $R_7$ zapojené v sérii rezistorem $R_{B7}$ podle vztahu: 
\[R_{B7} = R_B + R_7 = \num{195,852 190} + 355 = \SI{550,852 19}{\ohm}\]

%obvod 10
\begin{figure}[ht!]
\begin{center}
\begin{circuitikz}
    \draw
    (0,0) to[dcvsource, v^<=$U_v$] ++(0,4)
    to[R=$R_A$, -*] (2,4)
    to[short] (2,5)
    to[R=$R_{B7}$] (7,5)--(7,4);
    
    \draw
    (2,4) to[short] (2,3)
    to[R=$R_C$] (4.5,3)
    to [R=$R_6$] (7,3)
    to[short, -*] (7,4) -- (8,4)
    (8,4) -- (8,0)
    to[R=$R_8$] (1,0) -- (0,0);
     
\end{circuitikz}
\caption{Nahrazení rezistorů $R_B$ a $R_7$ rezistorem $R_{B7}$}
\end{center}
\end{figure}

\noindent
Tento postup použijeme také pro rezistory $R_C$ a $R_6$, které jsou opět zapojené v sérii. 
Nahradíme je \newline rezistorem $R_{C6}$ podle vztahu: 
\[R_{C6} = R_C + R_6 = \num{213,537 502} + 870 = \SI{1083,537 502}{\ohm}\]

%obvod 11
\begin{figure}[ht!]
\begin{center}
\begin{circuitikz}
    \draw
    (0,0) to[dcvsource, v^<=$U_v$] ++(0,4)
    to[R=$R_A$, -*] (2,4)
    to[short] (2,5)
    to[R=$R_{B7}$] (5,5)--(5,4);
    
    \draw
    (2,4) to[short] (2,3)
    to [R=$R_{C6}$] (5,3)
    to[short, -*] (5,4) -- (6,4)
    (6,4) -- (6,0)
    to[R=$R_8$] (1,0) -- (0,0);
     
\end{circuitikz}
\caption{Nahrazení rezistorů $R_C$ a $R_6$ rezistorem $R_{C6}$}
\end{center}
\end{figure}

\newpage

\noindent
Nyní si můžeme povšimnout, že nově vzniklé rezistory $R_{B7}$ a $R_{C6}$ jsou zapojené paralerně. 
Nahradíme je rezistorem $R_{BC67}$ podle vztahu:

\[R_{BC67} = \frac{R_{B7} * R_{C6}}{R_{B7} + R_{C6}} = \frac{\num{550,852 19} * \num{1083,537 502}}{\num{550,852 19} + \num{1083,537 502}} = \SI{365,193 814}{\ohm}\]

%obvod 12
\begin{figure}[ht!]
\begin{center}
\begin{circuitikz}
    \draw
    (0,0) to[dcvsource, v^<=$U_v$] ++(0,4)
    to[R=$R_A$] (3,4)
    to[R=$R_{BC67}$] (6,4) -- (6,0)
    to[R=$R_8$] (0,0);
     
\end{circuitikz}
\caption{Nahrazení rezistorů $R_{B7}$ a $R_{C6}$ rezistorem $R_{BC67}$}
\end{center}
\end{figure}

\noindent
Dalším postupem ke zjednodušení obvodu je nahradit rezistory $R_A$ a $R_{BC67}$, zapojené v sérii, 
rezistorem $R_{ABC67}$ dle vztahu:
\[R_{ABC67} = R_A + R_{BC67} = \num{252,531 306} + \num{365,193 814} = \SI{617,725 12}{\ohm}\]

%obvod 13
\begin{figure}[ht!]
\begin{center}
\begin{circuitikz}
    \draw
    (0,0) to[dcvsource, v^<=$U_v$] ++(0,4)
    to[R=$R_{ABC67}$] (5,4) -- (5,0)
    to[R=$R_8$] (0,0);
     
\end{circuitikz}
\caption{Nahrazení rezistorů $R_A$ a $R_{BC67}$ rezistorem $R_{ABC67}$}
\end{center}
\end{figure}

\newpage

\noindent
Posledním krokem je již nahradit rezistory $R_{ABC67}$ a $R_8$ zapojené v sérii výsledným rezistorem $R_{EKV}$ podle vztahu:
\[R_{EKV} = R_{ABC67} + R_8 = \num{617,725 12} + 265 = \SI{882,725 12}{\ohm}\]

%obvod 14
\begin{figure}[ht!]
\begin{center}
\begin{circuitikz}
    \draw
    (0,0) to[dcvsource, v^<=$U_v$] ++(0,4)
    to[short] (4,4)
    to[R=$R_{EKV}$, i>_=$I$] (4,0) -- (0,0);
     
\end{circuitikz}
\caption{Nahrazení rezistorů $R_{ABC67}$ a $R_8$ rezistorem $R_{EKV}$}
\end{center}
\end{figure}

\noindent
Po zjednodušení obvodu můžeme vypočítat hodnotu proudu $I$ dle Ohmova zákona:
\[I = \frac{U_V}{R} = \frac{215}{\num{882,725 12}} = \SI{0,243 564}{\ampere}\]

\vspace{1cm}
\subsection{Zpětné počítání hodnot napětí a proudů}
Nyní máme obvod zjednodušený a můžeme zpětně dopočítat jednotlivé hodnoty napětí a proudů.
V následujících výpočtech opět využijeme Ohmův zákon: 
\[I = \frac{U}{R}\]
Pro kontrolu můžeme využít také Kirchhoffových zákonů.
\vspace{7mm}
\newline
Z následujícího obvodu nám vyplývají vztahy pro napětí $U_{R_{ABC67}}$ a $U_{R_8}$:
\[U_{R_{ABC67}} = I * R_{ABC67} = \num{0,243 564} * \num{617,725 12} = \SI{150,455 601}{\volt}\]
\[U_{R_8} = I * R_8 = \num{0,243 564} * 265 = \SI{64,54446}{\volt}\]

%obvod 15
\begin{figure}[ht!]
\begin{center}
\begin{circuitikz}
    \draw
    (0,0) to[dcvsource, v^<=$U_v$] ++(0,4)
    to[R=$R_{ABC67}$, v=$U_{R_{ABC67}}$, i>^=$I$] (5,4) -- (5,0)
    to[R=$R_8$, v=$U_{R_8}$] (0,0);
     
\end{circuitikz}
\caption{Napětí $U_{R_{ABC67}}$ a $U_{R_8}$:}
\end{center}
\end{figure}

\noindent
Pro kontrolu využijeme \rom{2}. Kirchhoffův zákon a to v následujcím tvaru:
\[U_V - U_{R_{ABC67}} - U_{R_8} = 0\]
\[215 - \num{150,455 601} - \num{64,544 46} \approx 0\]

\noindent
Z dalšího obvodu nám vyplývají vztahy pro napětí $U_{R_A}$ a $U_{R_{BC67}}$:
\[U_{R_A} = I * R_A = \num{0,243 564} * \num{252,531 306} = \SI{64,507 535}{\volt}\]
\[U_{R_{BC67}} = I * R_{BC67} = \num{0,243 564} * \num{365,193 814} = \SI{88,948 066}{\volt}\]


%obvod 16
\begin{figure}[ht!]
\begin{center}
\begin{circuitikz}
    \draw
    (0,0) to[dcvsource, v^<=$U_v$] ++(0,4)
    to[R=$R_A$, v=$U_{R_A}$, i>^=$I$] (3,4)
    to[R=$R_{BC67}$, v=$U_{R_{BC67}}$] (6,4) -- (6,0)
    to[R=$R_8$, v=$U_{R_8}$] (0,0);
     
\end{circuitikz}
\caption{Napětí $U_{R_A}$ a $U_{R_{BC67}}$}
\end{center}
\end{figure}

\noindent
Kontrola pomocí \rom{2}. Kirchhoffova zákona:
\[U_V - U_{R_A} - U_{R_{BC67}} - U_{R_8} = 0\]
\[215 - \num{64,507 535} - \num{88,948 066} - 265 \approx 0\]

\newpage

\noindent
Nyní si můžeme vyjádřit vztahy pro proudy $I_{R_{B7}}$ a $I_{R_{C6}}$:

\[I_{R_{B7}} = \frac{U_{R_{BC67}}}{R_{B7}} = \frac{\num{88,948 066}}{\num{550,852 19}} = \SI{0,161474}{\ampere}\]
\[I_{R_{C6}} = \frac{U_{R_{BC67}}}{R_{C6}} = \frac{\num{88,948 066}}{\num{1 083,537 502}} = \SI{0,082090436}{\ampere}\footnote{Výsledek uveden na 9 desetinných míst za účelem přesnosti následného výpočtu $U_{R_6}$.}\]

%obvod 17
\begin{figure}[ht!]
\begin{center}
\begin{circuitikz}
    \draw
    (0,0) to[dcvsource, v^<=$U_v$] ++(0,4)
    to[R=$R_A$, -*, v=$U_{R_A}$, i>^=$I$] (2,4)
    to[short] (2,5)
    to[R=$R_{B7}$, i>^=$I_{R_{B7}}$] (5,5)--(5,4);
    
    \draw
    (2,4) to[short] (2,3)
    to [R=$R_{C6}$, v=$U_{R_{BC67}}$, i>^=$I_{R_{C6}}$] (5,3)
    to[short, -*] (5,4) -- (6,4)
    (6,4) -- (6,0)
    to[R=$R_8$, v=$U_{R_8}$] (1,0) -- (0,0);
     
\end{circuitikz}
\caption{Proudy $I_{R_{B7}}$ a $I_{R_{C6}}$}
\end{center}
\end{figure}

\noindent
Kontrola nyní podle \rom{1}. Kirchhoffova zákona:
\[I - I_{R_{B7}} - I_{R_{C6}} = 0\]
\[\num{0,243 564} - \num{0,161474} - \num{0,082090436} \approx 0\]

\newpage

\noindent
Jako poslední krok ve zpětném počítání hodnot napětí a proudů si vyjádříme vztahy pro napětí $U_{R_C}$ a námi 
hledané napětí $U_{R_6}$:

\[U_{R_C} = I_{R_{C6}} * R_C = \num{0,082090436} * \num{213,537 502} = \SI{17,529387}{\volt}\]
\[U_{R_6} = I_{R_{C6}} * R_6 = \num{0,082090436} * 870 = \SI{71,418679}{\volt}\]

%obvod 18
\begin{figure}[ht!]
\begin{center}
\begin{circuitikz}
    \draw
    (0,0) to[dcvsource, v^<=$U_v$] ++(0,4)
    to[R=$R_A$, -*, v=$U_{R_A}$, i>^=$I$] (2,4)
    to[short] (2,5)
    to[R=$R_{B7}$, v=$U_{R_{BC67}}$, i>^=$I_{R_{B7}}$] (7,5)--(7,4);
    
    \draw
    (2,4) to[short] (2,3)
    to[R=$R_C$, v=$U_{R_C}$, i>^=$I_{R_{C6}}$] (4.5,3)
    to [R=$R_6$, v=$U_{R_6}$] (7,3)
    to[short, -*] (7,4) -- (8,4)
    (8,4) -- (8,0)
    to[R=$R_8$, v=$U_{R_8}$] (1,0) -- (0,0);
     
\end{circuitikz}
\caption{Napětí $U_{R_C}$ a $U_{R_6}$}
\end{center}
\end{figure}

\noindent
Provedeme kontrolu pomocí \rom{2}. Kirchhoffova zákona:
\[U_V - U_{R_A} - U_{R_C} - U_{R_6} - U_{R_8} = 0\]
\[215 - \num{64,507 535} - \num{17,529387} - \num{71,418679} - \num{64,54446} \approx 0\]

\noindent
Výsledné hodnoty:
\[I_{R_6} = I_{R_{C6}} = \num{0,082090436} \doteq \doubleunderline{\SI{0,0821}{\ampere}}\]
\[U_{R_6} = \num{71,418679} \doteq \doubleunderline{\SI{71,4187}{\volt}}\]

\newpage

%Priklad 2
\rhead{\bfseries Příklad 2}

\section{Příklad 2 -- Théveninova věta}
\textbf{Zadání:} Stanovte napětí $U_{R_3}$ a proud $I_{R_3}$. Použijte metodu Théveninovy věty.

\begin{table}[ht]
  \begin{center}
    \begin{tabular}{|c|c|c|c|c|c|c|c|} 
      \hline
      %prvni radek
       sk. & $U$ [\si{\volt}] & $R_1$ [\si{\ohm}] & $R_2$ [\si{\ohm}]
       & $R_3$ [\si{\ohm}] & $R_4$ [\si{\ohm}] & $R_5$ [\si{\ohm}]
       & $R_6$ [\si{\ohm}]\\
       %druhy radek
       \hline
       E & 250 & 150 & 335 & 625 & 245 & 600 & 150 \\
     \hline
    \end{tabular}
    \caption{Zadané hodnoty}
    \label{tab:2}
  \end{center}
 \end{table}

%obvod 19
\begin{figure}[ht!]
\begin{center}
\begin{circuitikz}
    \draw
    (0,0) to[dcvsource, v^<=$U$] ++(0,2) -- (0,3)%zdroj U
    to[short, -*] (1,3)
    (1,3) -- (1,4)
    to[R=$R_1$, -*] (4,4) -- (5,4)
    to[R=$R_4$] (6,4) -- (7,4)
    to[R=$R_5$, -*] (7,2) -- (7,0)
    (7,0) -- (5,0)
    (5,0) -- (0,0);
    
    \draw
    (1,3) -- (1,2)
    to[R=$R_2$, -*] (4,2) -- (5,2)
    to[R=$R_6$] (6,2) -- (7,2);
    
    \draw
    (4,4) to[R=$R_3$, v=$U_{R_3}$, i>^=$I_{R_3}$] (4,2);
    
\end{circuitikz}
\caption{Zadaný obvod}
\end{center}
\end{figure}

\subsection{Vyjádření vztahu pro proud $I_{R_3}$}
Nejprve si nahradíme obvod bez rezistoru $R_3$ obvodem skutečného zdroje napětí:

%obvod 20
\begin{figure}[ht!]
\begin{center}
\begin{circuitikz}
    \draw
    (0,0) to[dcvsource, v^<=$U_I$] ++(0,3)%zdroj Ui
    to [R=$R_I$, -o] (4,3)
    to [R=$R_3$, i>^=$I_{R_3}$, -o] (4,0) -- (0,0);
    
\end{circuitikz}
\caption{Nahrazení části obvodu obvodem skutečného zdroje napětí}
\end{center}
\end{figure}

\noindent
Pomocí výše uvedeného obvodu si vyjádříme vztah pro $I_{R_3}$:

\[I_{R_3} = \frac{U_i}{R_i + R_3}\]

\vspace{1cm}
\noindent
Dále budeme pokračovat výpočtem prozatím neznámých hodnot, a to hodnot $U_i$ a $R_i$.

\newpage
\subsection{Výpočet hodnot odporu $R_i$ a napětí $U_i$}
Nyní překreslíme obvod bez rezistoru $R_3$ a napěťový zdroj nahradíme zkratem:

%obvod 21
\begin{figure}[ht!]
\begin{center}
\begin{circuitikz}
    \draw
    (0,0) -- (0,2)
    (0,2) -- (0,3)
    to[short, -*] (1,3)
    (1,3) -- (1,4)
    to[R=$R_1$, -o] (4,4) -- (5,4)
    to[R=$R_4$] (6,4) -- (7,4)
    to[R=$R_5$, -*] (7,2) -- (7,0)
    (7,0) -- (5,0)
    (5,0) -- (0,0);
    
    \draw
    (1,3) -- (1,2)
    to[R=$R_2$, -o] (4,2) -- (5,2)
    to[R=$R_6$] (6,2) -- (7,2);
    
\end{circuitikz}
\caption{Obvod bez rezistoru $R_3$ s nahrazením napěťového zdroje zkratem}
\end{center}
\end{figure}

\noindent
Následně si obvod postupně zjednodušíme:

\[R_{45} = R_4 + R_5\]

%obvod 22
\begin{figure}[ht!]
\begin{center}
\begin{circuitikz}
    \draw
    (0,0) -- (0,2)
    (0,2) -- (0,3)
    to[short, -*] (1,3)
    (1,3) -- (1,4)
    to[R=$R_1$, -o] (4,4) -- (5,4)
    to[R=$R_{45}$] (6,4) -- (7,4)
    (7,4) -- (7,0)
    (7,0) -- (5,0)
    (5,0) -- (0,0);
    
    \draw
    (1,3) -- (1,2)
    to[R=$R_2$, -o] (4,2) -- (5,2)
    to[R=$R_6$] (6,2) -- (7,2);
    
\end{circuitikz}
\caption{Nahrazení rezistorů $R_4$ a $R_5$ rezistorem $R_{45}$}
\end{center}
\end{figure}

\newpage
\noindent
Obvod si překreslíme pro lepší názornost:

%obvod 23
\begin{figure}[ht!]
\begin{center}
\begin{circuitikz}
    \draw
    (4,6) to[short, -o] (4,6) -- (4,5)
    (4,5) -- (3,5)
    to[R=$R_1$, -*] (3,3)
    to[R=$R_2$] (3,1) -- (4,1)
    to [short, -o] (4,0);
    
    \draw
    (4,5) -- (5,5)
    to[R=$R_{45}$, -*] (5,3)
    to[R=$R_6$] (5,1) -- (4,1);
    
    \draw
    (3,3) -- (5,3);
    
\end{circuitikz}
\caption{Překreslený obvod 1}
\end{center}
\end{figure}

\noindent
Dále si obvod překreslíme tak, aby lépe vynikly zapojení mezi jednotlivými rezistory:

%obvod 24
\begin{figure}[ht!]
\begin{center}
\begin{circuitikz}
    \draw
    (4,7) to[short, -o] (4,7) -- (4,6)
    (4,6) -- (3,6)
    to[R=$R_1$] (3,4)
    to[short, -*] (4,4)
    to[short, -*] (4,3) -- (3,3)
    to[R=$R_2$] (3,1) -- (4,1)
    to [short, -o] (4,0);
    
    \draw
    (4,6) -- (5,6)
    to[R=$R_{45}$] (5,4) -- (4,4);
    
    \draw
    (4,3) -- (5,3)
    to[R=$R_6$] (5,1) -- (4,1);
    
\end{circuitikz}
\caption{Překreslený obvod 2}
\end{center}
\end{figure}

\newpage
\noindent
Obvod postupně dále zjednodušíme:

%obvod 25
\begin{figure}[ht!]
\begin{center}
\begin{circuitikz}
    \draw
    (4,7) to[short, -o] (4,7) -- (4,6)
    (4,6) -- (3,6)
    to[R=$R_1$] (3,4)
    to[short, -*] (4,4)
    to[short, -*] (4,3) -- (3,3)
    to[R=$R_2$] (3,1) -- (4,1)
    to [short, -o] (4,0);
    
    \draw
    (4,6) -- (5,6)
    to[R=$R_{45}$] (5,4) -- (4,4);
    
    \draw
    (4,3) -- (5,3)
    to[R=$R_6$] (5,1) -- (4,1);
    
    \draw
    [-{Latex[length=5mm, width=2mm]}] (6,3.5)--(8,3.5);
    
    \draw
    (9,7) to[short, -o] (9,7) -- (9,6)
    to[R=$R_{145}$] (9,4) -- (9,3)
    to[R=$R_{26}$] (9,1)
    to [short, -o] (9,0);
    
    \draw
    [-{Latex[length=5mm, width=2mm]}] (10,3.5)--(12,3.5);
    
    \draw
    (13,7) to[short, -o] (13,7)
    to[R=$R_i$, -o] (13,0);
    
\end{circuitikz}
\caption{Postupné zjednodušení obvodu}
\end{center}
\end{figure}

\noindent
Ze zmíněného zjednodušení obvodu vyplývají následující vztahy mezi rezistory:

\[R_{145} = \frac{R_1 * R_{45}}{R_1 + R_{45}}\]
\[R_{26} = \frac{R_2 * R_6}{R_2 + R_6}\]
\[R_i = R_{145} + R_{26}\]

\noindent
Vypočítáme hodnotu rezistoru $R_i$:

\[R_i = \frac{R_1 * (R_4 + R_5)}{R_1 + R_4 + R_5} + \frac{R_2 * R_6}{R_2 + R_6} = \frac{150 * (245 + 600)}{150 + 245 + 500} + \frac{335 * 150}{335 + 150} =  \SI{230,995182}{\ohm}\]

\noindent
Dále překreslíme obvod bez $R_3$ a určíme napětí naprázdno:

%obvod 26
\begin{figure}[ht!]
\begin{center}
\begin{circuitikz}
    \draw
    (0,0) to[dcvsource, v^<=$U$] ++(0,2) -- (0,3)%zdroj U
    to[short, -*] (1,3)
    (1,3) -- (1,4)
    to[R=$R_1$, -o, v=$U_{R_1}$] (4,4) -- (5,4)
    to[R=$R_4$] (6,4) -- (7,4)
    to[R=$R_5$, -*] (7,2) -- (7,0)
    (7,0) -- (5,0)
    (5,0) -- (0,0);
    
    \draw
    (1,3) -- (1,2)
    to[R=$R_2$, -o, v=$U_{R_2}$] (4,2) -- (5,2)
    to[R=$R_6$] (6,2) -- (7,2);
    
    \draw
    (4,4) to[open, v=$U_i$] (4,2);
    
    \draw
    [thin,<-,=triangle 45] (2.5,2.9)node{\rom{1}.}  ++(-60:0.3) arc (-60:170:0.3);
    
\end{circuitikz}
\caption{Obvod bez $R_3$}
\end{center}
\end{figure}

\noindent
Pomocí \rom{2}. Kirchhoffova zákona si vyjádříme vztah napětí ve smyčce vyznačené v obvodu:
\vfill
\noindent
\rom{1}.
\[U_i + U_{R_1} - U_{R_2} = 0\]
\[U_i = U_{R_2} - U_{R_1}\]

\noindent
Dále si vyjádříme vztahy napětí $U_{R_1}$ a napětí $U_{R_2}$:

\[U_{R_1} = U * \frac{R_1}{R_1 + R_4 + R_5}\]
\[U_{R_2} = U * \frac{R_2}{R_2 + R_6}\]

\noindent
Výše zmíněné vztahy si dosadíme do vzorce pro výpočet napětí $U_i$. 
Do nově vzniklého vztahu dosadíme číselné hodnoty a vypočítáme hodnotu napětí $U_i$: 

\[U_i = U_{R_2} - U_{R_1}\]
\[U_i = U * \frac{R_2}{R_2 + R_6} - U * \frac{R_1}{R_1 + R_4 + R_5}\]
\[U_i = U * \left(\frac{R_2}{R_2 + R_6} - \frac{R_1}{R_1 + R_4 + R_5} \right)\]
\[U_i = 250 * \left(\frac{335}{335 + 150} - \frac{150}{150 + 245 + 600} \right)\]
\[U_i = \SI{134,99197}{\volt}\]

\subsection{Výpočet hodnot proudu $I_{R_3}$ a napětí $U_{R_3}$}
Nyní máme vypočítané hodnoty pro napětí $U_i$ a odpor $R_i$. Můžeme tedy vypočítat hodnoty proudu $I_{R_3}$ \newline a napětí $U_{R_3}$:

\[I_{R_3} = \frac{U_i}{R_i + R_3} = \frac{\num{134,991 97}}{\num{230,995 182} + 625} = \num{0,157 701 788}  \doteq \doubleunderline{\SI{0,1577}{\ampere}}\]

\[U_{R_3} = R_3 * I_{R_3} = 625 * \num{0,157 701 788} = \num{98,563 618} \doteq \doubleunderline{\SI{98,5636}{\volt}}\]
\newpage

%Priklad 3
\rhead{\bfseries Příklad 3}

\section{Příklad 3 -- Metoda uzlových napětí}
\textbf{Zadání:} Stanovte napětí $U_{R_2}$ a proud $I_{R_2}$. Použijte metodu uzlových napětí ($U_A$, $U_B$, $U_C$).

\begin{table}[ht]
  \begin{center}
    \begin{tabular}{|c|c|c|c|c|c|c|c|c|} 
      \hline
      %prvni radek
       sk. & $U$ [\si{\volt}] & $I_1$ [\si{\ampere}] & $I_2$ [\si{\ampere}]
       & $R_1$ [\si{\ohm}] & $R_2$ [\si{\ohm}] & $R_3$ [\si{\ohm}]
       & $R_4$ [\si{\ohm}] & $R_5$ [\si{\ohm}]\\
       %druhy radek
       \hline
       D & 115 & 0,6 & 0,9 & 50 & 38 & 48 & 37 & 28\\
     \hline
    \end{tabular}
    \caption{Zadané hodnoty}
    \label{tab:3}
  \end{center}
 \end{table}

%obvod 27
\begin{figure}[ht!]
\begin{center}
\begin{circuitikz}
     \draw
    (0,0) to[dcvsource, v^<=$U$] ++(0,6)%zdroj U
    to[R=$R_1$, -*] (3,6)
    to[R=$R_3$, -*] (8,6) -- (11,6)
    (11,6) -- (11,4)
    to[ioosource, v^=$I_1$] (11,2) -- (11,0)
    to[short, -*] (8,0)
    to[R=$R_4$, -*] (3,0) -- (0,0);
    
    \draw
    (3,6) -- (3,4)
    to[R=$R_2$, v=$U_{R_2}$, i>^=$I_{R_2}$] (3,2)
    to[short, -*]  (3,0) -- (3,-3)
    (3,-3) -- (4.5, -3)
    to[ioosource, v>=$I_2$] (6.5,-3) -- (8,-3)
    (8,-3) -- (8,0); 
    
    \draw
    (8,6) to [R=$R_5$, -*] (8,0);
    
    \draw
    node (A) at (8,6) {}
    node (B) at (3.25,0.25) {}
    [thick, -latex] (A) -- (B) node[midway,above left] {$U_B$};
    
    \draw
    node (A) at (3,6) {}
    node (B) at (3,0) {}
    [thick, -latex] (A) to[bend left] (B);
    
    \draw
    node (A) at (8,0) {}
    node (B) at (3,0) {}
    [thick, -latex] (A) to[bend left] (B);
    
    \draw
    node (A) at (4.25,4) {$U_A$}
    node (B) at (5.5,-1.2) {$U_C$};
    
\end{circuitikz}
\caption{Zadaný obvod}
\end{center}
\end{figure}

\newpage
\subsection{Vyjádření vztahů mezi jednotlivými proudy obvodu}
Jednotlivé vztahy mezi proudy v obvodu vyjádříme pomocí \rom{1}. Kirchhoffova zákona pro uzly A, B a D. 
Uzel C bude pro nás sloužit jako referenční uzel:

\[A: I_{R_1} + I_{R_3} - I_{R_2} = 0\]
\[B: I_1 - I_{R_3} - I_{R_5} = 0\]
\[D: I_2 + I_{R_5} - I_{R_4} - I_1 = 0\]

%obvod 28
\begin{figure}[ht!]
\begin{center}
\begin{circuitikz}
     \draw
    (0,0) to[dcvsource, v^<=$U$] ++(0,6)%zdroj U
    to[R=$R_1$, -*, i^=$I_{R_1}$] (3,6)
    node[label={above:\textcolor{red}{A}}] {}
    to[R=$R_3$, -*, i<^=$I_{R_3}$] (8,6)
    node[label={above:\textcolor{red}{B}}] {}
    (8,6) -- (11,6)
    (11,6) -- (11,4)
    to[ioosource, v^=$I_1$] (11,2) -- (11,0)
    to[short, -*] (8,0)
    node[label={below:\textcolor{red}{D}}, xshift=0.25cm] {}
    to[R=$R_4$, -*,  i>_=$I_{R_4}$] (3,0)
    node[label={below:\textcolor{red}{C}},  xshift=0.2cm] {}
    (3,0) -- (0,0);
    
    \draw
    (3,6) -- (3,4)
    to[R=$R_2$, v=$U_{R_2}$, i>^=$I_{R_2}$] (3,2)
    to[short, -*]  (3,0) -- (3,-3)
    (3,-3) -- (4.5, -3)
    to[ioosource, v>=$I_2$] (6.5,-3) -- (8,-3)
    (8,-3) -- (8,0); 
    
    \draw
    (8,6) -- (8,4)
    to [R=$R_5$, i>^=$I_{R_5}$] (8,2)
    to[short, -*] (8,0);
    
    \draw
    node (A) at (8,6) {}
    node (B) at (3.25,0.25) {}
    [thick, -latex] (A) -- (B) node[midway,above left] {$U_B$};
    
    \draw
    node (A) at (3,6) {}
    node (B) at (3,0) {}
    [thick, -latex] (A) to[bend left] (B);
    
    \draw
    node (A) at (8,0) {}
    node (B) at (3,0) {}
    [thick, -latex] (A) to[bend left] (B);
    
    \draw
    node (A) at (4.25,4) {$U_A$}
    node (B) at (5.5,-1.2) {$U_C$};
    
\end{circuitikz}
\caption{Zadaný obvod s označemýni uzly a proudy}
\end{center}
\end{figure}

\newpage
\subsection{Vytvoření náhradních obvodů}
Dále si vytvoříme náhradní obvody pro určení vztahů pro jednotlivé proudy obvodu.
\vspace{10mm}
\newline
Určení proudu $I_{R_1}$:

\[R_1 * I_{R_1} + U_A - U = 0\]
\[I_{R_1} = \frac{U - U_A}{R_1}\]

%obvod 29
\begin{figure}[ht!]
\begin{center}
\begin{circuitikz}
    \draw
    (0,0) to[dcvsource, v^<=$U$] ++(0,3)%zdroj Ui
    to [R=$R_1$, i>^=$I_{R_1}$, -o] (4,3)
    node[label={above:\textcolor{red}{A}}] {}
    to [open, v=$U_A$, -o] (4,0)
    node[label={below:\textcolor{red}{C}}] {}
    (4,0) -- (0,0);
    
    \draw
    [thin,<-,=triangle 45] (2,1.5)node{}  ++(-60:0.5) arc (-60:170:0.5);
    
\end{circuitikz}
\caption{Náhradní obvod pro určení proudu $I_{R_1}$}
\end{center}
\end{figure}

\noindent
Určení proudu $I_{R_2}$:
\[R_2 * I_{R_2} - U_A = 0\]
\[I_{R_2} = \frac{U_A}{R_2}\]

%obvod 30
\begin{figure}[ht!]
\begin{center}
\begin{circuitikz}
     \draw
    (0,0) to[short, -o] (0,0)
    node[label={below:\textcolor{red}{C}}] {}
    to[R=$R_2$, i<=$I_{R_2}$, -o] (0,4)
    node[label={above:\textcolor{red}{A}}] {};
    
     \draw
    node (A) at (0,4) {}
    node (B) at (0,0) {}
    [thick, -latex] (A) to[bend left] (B);
    
    \draw
    node (A) at (1.2,2) {$U_A$};
    
    \draw
    [thin,->,=triangle 45] (0.42,2)node{}  ++(-60:0.15) arc (-60:170:0.15);
    
\end{circuitikz}
\caption{Náhradní obvod pro určení proudu $I_{R_2}$}
\end{center}
\end{figure}

\newpage
\noindent
Určení proudu $I_{R_3}$:

\[R_3 * I_{R_3} + U_A - U_B = 0\]
\[I_{R_3} = \frac{U_B - U_A}{R_3}\]

%obvod 31
\begin{figure}[ht!]
\begin{center}
\begin{circuitikz}
    \draw
    (0,0) to[short, -o] (0,0)
    node[label={below:\textcolor{red}{C}}] {}
    to[open, -o] (0,3)
    node[label={above:\textcolor{red}{A}}] {}
    to [R=$R_3$, i<^=$I_{R_3}$, -o] (4,3)
    node[label={above:\textcolor{red}{B}}] {};
    
    \draw
    [thin,->,=triangle 45] (0.7,1.5)node{}  ++(-60:0.5) arc (-60:170:0.5);
    
    \draw
    node (A) at (0,3) {}
    node (B) at (0,0) {}
    [thick, -latex] (A) to[bend right] (B);
    
    \draw
    node (A) at (-1,1.5) {$U_A$};
    
     \draw
    node (A) at (4,3) {}
    node (B) at (0,0) {}
    [thick, -latex] (A) -- (B) node[midway,below right] {$U_B$};
    
\end{circuitikz}
\caption{Náhradní obvod pro určení proudu $I_{R_3}$}
\end{center}
\end{figure}

\noindent
Určení proudu $I_{R_4}$:

\[R_4 * I_{R_4} - U_C = 0\]
\[I_{R_4} = \frac{U_C}{R_4}\]

%obvod 32
\begin{figure}[ht!]
\begin{center}
\begin{circuitikz}
     \draw
    (0,0) to[short, -o] (0,0)
    node[label={above:\textcolor{red}{C}}] {}
    to[R=$R_4$, i<=$I_{R_4}$, -o] (4,0)
    node[label={above:\textcolor{red}{D}}] {};
    
     \draw
    node (A) at (4,0) {}
    node (B) at (0,0) {}
    [thick, -latex] (A) to[bend left] (B);
    
    \draw
    node (A) at (2,-1) {$U_C$};
    
    \draw
    [thin,->,=triangle 45] (2,-0.42)node{}  ++(-60:0.15) arc (-60:170:0.15);
    
\end{circuitikz}
\caption{Náhradní obvod pro určení proudu $I_{R_4}$}
\end{center}
\end{figure}

\newpage
\noindent
Určení proudu $I_{R_5}$:

\[R_5 * I_{R_5} + R_4 * I_{R_4} - U_B = 0\]
\[I_{R_5} = \frac{U_B - R_4 * I_{R_4}}{R_5} = \frac{U_B - R_4 * \frac{U_C}{R_4}}{R_5}\]
\[I_{R_5} = \frac{U_B - U_C}{R_5}\]

%obvod 33
\begin{figure}[ht!]
\begin{center}
\begin{circuitikz}
    \draw
    (0,0) to[short, -o] (0,0)
    node[label={below:\textcolor{red}{C}}] {}
    to[R=$R_4$, i<^=$I_{R_4}$, -o] (4,0)
    node[label={below:\textcolor{red}{D}}] {}
    to [R=$R_5$, i<^=$I_{R_5}$, -o] (4,4)
    node[label={above:\textcolor{red}{B}}] {};
    
    \draw
    [thin,<-,=triangle 45] (2.2,1.2)node{}  ++(-60:0.5) arc (-60:170:0.5);
    
     \draw
    node (A) at (4,4) {}
    node (B) at (0,0) {}
    [thick, -latex] (A) -- (B) node[midway,above left] {$U_B$};
    
\end{circuitikz}
\caption{Náhradní obvod pro určení proudu $I_{R_3}$}
\end{center}
\end{figure}

\subsection{Výpočet hodnoty napětí $U_{R_2}$ a proudu $I_{R_2}$}

Nejprve dosadíme do vyjádřených vztahů mezi proudu vztahy pro jednotlivé proudy:

\[A:  \frac{U - U_A}{R_1} + \frac{U_B - U_A}{R_3} - \frac{U_A}{R_2} = 0\]
\[B: I_1 - \frac{U_B - U_A}{R_3} - \frac{U_B - U_C}{R_5} = 0\]
\[C: I_2 + \frac{U_B - U_C}{R_5} - \frac{U_C}{R_4} - I_1 = 0\]

\noindent
Upravíme rovnice:

\[R_2 * R_3 * (U - U_A) + R_1 * R_2 * (U_B - U_A) - R_1 * R_3 * U_A = 0\]
\[R_3 * R_5 * I_1 - R_5 * (U_B - U_A) - R_3 * (U_B - U_C) = 0\]
\[R_4 * R_5 * I_2 + R_4 * (U_B - U_C) - R_5 * U_C - R_4 * R_5 * I_1 = 0\]

\noindent
Dosadíme číselné hodnoty a upravíme rovnice na potřebný tvar:

\[38 * 48 * (115 - U_A) + 50 * 38 * (U_B - U_A) - 50 * 48 * U_A = 0\]
\[48 * 28 * 0,6 - 28 * (U_B - U_A) - 48 * (U_B - U_C) = 0\]
\[37 * 28 * 0,9 + 37 * (U_B - U_C) - 28 * U_C - 37 * 28 * 0,6 = 0\]

\begin{center}
    \line(1,0){350}
\end{center}

\[\num{209 760} - \num{1 824} * U_A + \num{1 900} * U_B - \num{1 900} * U_A - \num{2 400} * U_A = 0\]
\[806,4 - 28 * U_B + 28 * U_A - 48 * U_B + 48 * U_C = 0\]
\[932,4 + 37 * U_B - 37 * U_C - 28 * U_C - 621,6 = 0\]

\begin{center}
    \line(1,0){350}
\end{center}

\newpage

\[\num{-6 124} * U_A + \num{1 900} * U_B + \num{209 760} = 0 \hspace{5mm} /:8\]
\[28 * U_A - 76 * U_B + 48 * U_C + 806,4 = 0 \hspace{5mm} /:4\]
\[37 * U_B - 65 * U_C + 370,8 = 0\]

\begin{center}
    \line(1,0){350}
\end{center}

\[-765,5 * U_A + 237,5 * U_B = \num{-26 220}\]
\[7 * U_A - 19 * U_B + 12 * U_C = -201,6\]
\[37 * U_B - 65 * U_C = -310,8\]

\vspace{1cm}
\noindent
Získali jsme tři rovnice o třech neznámých. Nyní si sestavíme maticovou rovnici:

\begin{equation*}
\begin{pmatrix}
-765,5 & 237,5 & 0 \\
7 & -19 & 12 \\
0 & 37 & -65
\end{pmatrix}
*
\begin{pmatrix}
U_A \\
U_B \\
U_C
\end{pmatrix}
=
\begin{pmatrix}
\num{-26 220} \\
-201,6 \\
-310,8
\end{pmatrix}
\end{equation*}

\noindent
Dále si určíme hodnoty determinantů:

\begin{equation*}
|D| = 
   \begin{vmatrix} 
   -765,5 & 237,5 & 0  \\
   7 & -19 & 12  \\
   0 & 37 & -65  \\
   \end{vmatrix} 
=   
\end{equation*}
$= [-765.5 * (-19) * (-65)] + (7 * 37 * 0) + (0 * 237,5 * 12) -\\  
- [0 * (-19) * 0] - [12 * 37 * (-765,5)] - (-65 * 237,5 * 7) =\\
= \num{-945 392,5} + 0 + 0 + 0 + \num{339 882} + \num{108 062,5} =\\
= \num{-497 448}$

\begin{equation*}
|D_1| = 
   \begin{vmatrix} 
   \num{-26 220} & 237,5 & 0  \\
   -201,6 & -19 & 12  \\
   -310,8 & 37 & -65  \\
   \end{vmatrix} 
=  
\end{equation*}
$= [\num{-26 220} * (-19) * (-65)] + (-201,6 * 37 * 0) + (-310,8 * 237,5 * 12) -\\
- [0 * (-19) * (-310,8)] - [12 * 37 * (\num{-26 220})] - [-65 * 237,5 * (-201,6)] =\\
= -\num{32 381 700} - \num{885 780} + \num{11 641 680} - \num{3 112 200} =\\
= \num{-24 738 000}$ 

\vspace{1cm}
\noindent
Vypočítáme si hodnotu napětí $U_A$:

\[U_A = \frac{|D_1|}{|D|} = \frac{\num{-24 738 000}}{\num{-497 448}} = \SI{49,729821}{\volt}\]

\noindent
Hodnota napětí $U_{R_2}$ je stejná, jako hodnota napětí $U_A$, tedy:

\[U_{R_2} = U_A = \SI{49,729 821}{\volt} \doteq \doubleunderline{\SI{49,729 8}{\volt}}\]

\noindent
Hodnotu proudu $I_{R_2}$ vypočítáme pomocí Ohmova zákona:

\[I_{R_2} = \frac{U_A}{R_2} = \frac{\num{49,729 821}}{38} = \num{1,308 68} \doteq \doubleunderline{\SI{1,3087}{\ampere}}\]
\newpage

%Priklad 4
\rhead{\bfseries Příklad 4}

\section{Příklad 4 -- Metoda smyčkových proudů}
\textbf{Zadání:} Pro napájecí napětí platí: $u_1 = U_1 * \sin{(2\pi ft)}$, $u_2 = U_2 * \sin{(2\pi ft)}$. \newline
Ve vztahu pro napětí $u_{L_2} = U_{L_2} * \sin{(2\pi ft + \varphi_{L_2})}$ určete $|U_{L_2}|$ a $\varphi_{L_2}$. 
Použijte metodu smyčkových proudů.
\vspace{0.5cm}\newline
Pozn: Pomocné směry šipek napájecích zdrojů platí pro speciání časový okamžik ($t = \frac{\pi}{2\omega}$).

\begin{table}[ht]
  \begin{center}
    \begin{tabular}{|c|c|c|c|c|c|c|c|c|c|} 
      \hline
      %prvni radek
       sk. & $U_1$ [\si{\volt}] & $U_2$ [\si{\volt}] & $R_1$ [\si{\ohm}]
       & $R_2$ [\si{\ohm}] & $L_1$ [\si{mH}] & $L_2$ [\si{mH}]
       & $C_1$ [\si{\mu F}] & $C_2$ [\si{\mu F}] & $f$ [\si{Hz}]\\
       %druhy radek
       \hline
       H & 65 & 60 & 10 & 10 & 160 & 75 & 155 & 70 & 95\\
     \hline
    \end{tabular}
    \caption{Zadané hodnoty}
    \label{tab:4}
  \end{center}
 \end{table}

%obvod 34
\begin{figure}[ht!]
\begin{center}
\begin{circuitikz}
    \draw
    (0,0) to[short, -*] (0,3)
    to[R=$R_1$] (0,6)
    to[vsourcesin, v^=$u_1$] (10,6) 
    to[short, -*] (10,3)
    to[vsourcesin, v=$u_2$] (10,0); 
    
    \draw
    (0,0) to[cute inductor, l=$L_1$, -*] (5,0)
    to[C, l=$C_2$] (10,0);
    
    \draw
    (0,3) to[C, l=$C_1$, -*] (5,3)
    (5,3) to[cute inductor, l=$L_2$, i<=$i_{L_2}$, v<=$u_{L_2}$] (10,3);
    
    \draw
    (5,3) to[R=$R_2$] (5,0);
    
\end{circuitikz}
\caption{Zadaný obvod}
\end{center}
\end{figure}

\subsection{Výpočet úhlové frekvence}
Nejprve si spočítáme úhlovou frekvenci:
\[\omega = 2\pi f = 2 * 95 * \pi = 190\pi \frac{rad}{s}\]

\newpage
\subsection{Určení smyčkových proudů}
Dále si určíme smyčkové proudy a jejich směr:

%obvod 35
\begin{figure}[ht!]
\begin{center}
\begin{circuitikz}
    \draw
    (0,0) to[short, -*] (0,3)
    to[R=$R_1$] (0,6)
    to[vsourcesin, v^=$u_1$] (10,6) 
    to[short, -*] (10,3)
    to[vsourcesin, v=$u_2$] (10,0); 
    
    \draw
    (0,0) to[cute inductor, l=$L_1$, -*] (5,0)
    to[C, l=$C_2$] (10,0);
    
    \draw
    (0,3) to[C, l=$C_1$, -*] (5,3)
    (5,3) to[cute inductor, l=$L_2$, i<=$i_{L_2}$, v<=$u_{L_2}$] (10,3);
    
    \draw
    (5,3) to[R=$R_2$] (5,0);
    
    \draw
    [thin,<-,=triangle 45] (2.5,1.5)node{$i_B$}  ++(-60:0.5) arc (-60:170:0.5);
    
    \draw
    [thin,<-,=triangle 45] (7.5,1.5)node{$i_C$}  ++(-60:0.5) arc (-60:170:0.5);
    
    \draw
    [thin,<-,=triangle 45] (5,4.5)node{$i_A$}  ++(-60:0.5) arc (-60:170:0.5);
    
\end{circuitikz}
\caption{Obvod s vyznačenými smyčkovými proudy}
\end{center}
\end{figure}

\noindent
Poté sestavíme rovnice pro jednotlivé smyčky:
\[i_A: R_1 * I_A + Z_{L_2} * (I_A - I_C) + Z_{C_1} * (I_A - I_B) + U_1 = 0\]
\[i_B: Z_{C_1} * (I_B - I_A) + R_2 * (I_B - I_C) + Z_{L_1} * I_B = 0\]
\[i_C: Z_{C_2} * I_C + R_2 * (I_C - I_B) + Z_{L_2} * (I_C - I_A) + U_2 = 0\]

\subsection{Sestavení maticové rovnice}
Sestavíme maticovou rovnici:
\begin{equation*}
\begin{pmatrix}
R_1 + Z_{L_2} + Z_{C_1} & -Z_{C_1} & -Z_{L_2} \\
-Z_{C_1} & Z_{C_1} + R_2 + Z_{L_1} & -R_2 \\
-Z_{L_2} & -R_2 & Z_{C_2} + R_2 + Z_{L_2}
\end{pmatrix}
*
\begin{pmatrix}
I_A \\
I_B \\
I_C
\end{pmatrix}
=
\begin{pmatrix}
-U_1 \\
0 \\
-U_2
\end{pmatrix}
\end{equation*}
Dále dosadíme do maticové rovnice jednotlivé vztahy:

\[X_C = \frac{1}{\omega C}\]
\[X_L = \omega L\]

\[Z_C = \frac{1}{j\omega C} = \frac{1}{j\omega C} * \left(\frac{-j}{-j}\right) = \frac{-j}{\omega C} = -jX_C\]
\[Z_L = j\omega L = jX_L\]

\begin{equation*}
\begin{pmatrix}
R_1 + j\omega L_2 + \frac{-j}{\omega C_1} & -\frac{-j}{\omega C_1} & -j\omega L_2 \\
-\frac{-j}{\omega C_1} & \frac{-j}{\omega C_1} + R_2 + j\omega L_1 & -R_2 \\
-j\omega L_2 & -R_2 & \frac{-j}{\omega C_2} + R_2 + j\omega L_2
\end{pmatrix}
*
\begin{pmatrix}
I_A \\
I_B \\
I_C
\end{pmatrix}
=
\begin{pmatrix}
-U_1 \\
0 \\
-U_2
\end{pmatrix}
\end{equation*}

\begin{equation*}
\begin{pmatrix}
R_1 + j\omega L_2 - \frac{j}{\omega C_1} & \frac{j}{\omega C_1} & -j\omega L_2 \\
\frac{j}{\omega C_1} & \frac{-j}{\omega C_1} + R_2 + j\omega L_1 & -R_2 \\
-j\omega L_2 & -R_2 & \frac{-j}{\omega C_2} + R_2 + j\omega L_2
\end{pmatrix}
*
\begin{pmatrix}
I_A \\
I_B \\
I_C
\end{pmatrix}
=
\begin{pmatrix}
-U_1 \\
0 \\
-U_2
\end{pmatrix}
\end{equation*}
\newpage

\noindent
Poté dosadíme číselné hodnoty:
\begin{equation*}
\begin{pmatrix}
10 + \num{33,959 21}j & \num{10,808 485}j & \num{-44,767 695}j\\
\num{10,808 485}j & 10 + \num{84,695 932}j & -10\\
 \num{-44,767 695}j & -10 & 10 + \num{20,834 621}j
\end{pmatrix}
*
\begin{pmatrix}
I_A \\
I_B \\
I_C
\end{pmatrix}
=
\begin{pmatrix}
-65 \\
0 \\
-60
\end{pmatrix}
\end{equation*}

\vspace{1cm}
\subsection{Určení hodnot determinantů a proudů $I_A$ a $I_C$}
Nyní si určíme hodnoty jednotlivých determinantů:

\begin{equation*}
|D| = 
    \begin{vmatrix} 
        10 + \num{33,959 21}j & \num{10,808 485}j & \num{-44,767 695}j\\
        \num{10,808 485}j & 10 + \num{84,695 932}j & -10\\
        \num{-44,767 695}j & -10 & 10 + \num{20,834 621}j
    \end{vmatrix} 
=   
\end{equation*}
$= [( 10 + \num{33,959 21}j) * (10 + \num{84,695 932}j) * (10 + \num{20,834 621}j)] +\\
+ [(\num{10,808 485}j) * (-10) * (\num{-44,767 695}j)] +\\
+ [( \num{-44,767 695}j) * (\num{10,808 485}j) * (-10)] -\\  
- [(\num{-44,767 695}j) * (10 + \num{84,695 932}j) * (\num{-44,767 695}j)] -\\
- [(-10) * (-10) * (10 + \num{33,959 21}j)] -\\
- [(10 + \num{20,834 621}j) * (\num{10,808 485}j) * (\num{10,808 485}j)] =\\
= \underline{\num{-41 951,139 104} + \num{122 805,400 952}j}$

\vspace{1cm}
\begin{equation*}
|D_A| = 
    \begin{vmatrix} 
        -65 & \num{10,808 485}j & \num{-44,767 695}j\\
        0 & 10 + \num{84,695 932}j & -10\\
        -60 & -10 & 10 + \num{20,834 621}j
    \end{vmatrix} 
=   
\end{equation*}
$= [(-65) * (10 + \num{84,695 932}j) * (10 + \num{20,834 621}j)] +\\
+ [(0) * (-10) * (\num{-44,767 695}j)] +\\
+ [(-60) * (\num{10,808 485}j) * (-10)] -\\  
- [(\num{-44,767 695}j) * (10 + \num{84,695 932}j) * (-60)] -\\
- [(-10) * (-10) * (-65)] -\\
- [(10 + \num{20,834 621}j) * (\num{10,808 485}j) * (0)] =\\
= \underline{\num{342 197,995 913} - \num{88 970,385 45}j}$

\vspace{1cm}
\begin{equation*}
|D_C| = 
    \begin{vmatrix} 
        10 + \num{33,959 21}j & \num{10,808 485}j & -65\\
        \num{10,808 485}j & 10 + \num{84,695 932}j & 0\\
        \num{-44,767 695}j & -10 & -60
    \end{vmatrix} 
=   
\end{equation*}
$= [( 10 + \num{33,959 21}j) * (10 + \num{84,695 932}j) * (-60)] +\\
+ [(\num{10,808 485}j) * (-10) * (-65)] +\\
+ [( \num{-44,767 695}j) * (\num{10,808 485}j) * (0)] -\\  
- [(-65) * (10 + \num{84,695 932}j) * (\num{-44,767 695}j)] -\\
- [(0) * (-10) * (10 + \num{33,959 21}j)] -\\
- [(-60) * (\num{10,808 485}j) * (\num{10,808 485}j)] =\\
= \underline{\num{406 019,722 925} - \num{93 266,571 7}j}$

\[I_A = \frac{|D_A|}{|D|} = \frac{\num{342 197,995 913} - \num{88 970,385 45}j}{\num{-41 951,139 104} + \num{122 805,400 952}j} = \num{-1,501 19} - \num{2,273 69}j\si{\ampere}\]
\[I_C = \frac{|D_C|}{|D|} = \frac{\num{406 019,722 925} - \num{93 266,571 7}j}{\num{-41 951,139 104} + \num{122 805,400 952}j} = \num{-1,691 498} - \num{2,728 377}j\si{\ampere}\]

\newpage
\subsection{Výpočet $|U_{L_2}|$ a $\varphi_{L_2}$}
Nejprve vypočítáme hodnotu proudu $I_{L_2}$:
\[I_{L_2} = I_A - I_C = (\num{-1,501 19} - \num{2,273 69}j) - (\num{-1,691 498} - \num{2,728 377}j) = \num{0,190 308} + \num{0,454 687}j\si{\ampere}\]
\newline
Dále vypočítáme hodnotu napětí $U_{L_2}$:
\[U_{L_2} = I_{L_2} * Z_{L_2} = (\num{0,190 308} + \num{0,454 687}j) * (\num{44,767 695}j) = \num{-20,355 289} + \num{8,519 651}j\si{\volt}\]
\newline
Nyní již můžeme vypočítat hodnotu $|U_{L_2}|$ (amplituda napětí na cívce $L_2$):
\[|U_{L_2}| = \sqrt{(\num{-20,355 289})^2 + (\num{8,519 651})^2} = \num{22,066 314} \doteq \doubleunderline{\SI{22,0663}{\volt}}\]
\newline
Nakonec vypočítáme fázový posuv $\varphi_{L_2}$:
\[\varphi_{L_2} = \arctan{\left( \frac{im(U_{L_2})}{re(U_{L_2})}\right)} = \arctan{\left( \frac{\num{8,519 651}}{\num{-20,355 289}}\right)} = \num{-0,396 392} \doteq \doubleunderline{\SI{-0,3964}{rad}}\]
\newpage

%Priklad 5
\rhead{\bfseries Příklad 5}

\section{Příklad 5 -- Diferenciální rovnice}
\textbf{Zadání:} V obvodu na obrázku níže v čase $t = 0[s]$ sepne spínač S. Sestavte diferenciální rovnici popisující chování obvodu na obrázku, 
dále ji upravte dosazením hodnot parametrů. Vypočítejte analytické řešení $i_L = f(t)$. 
Proveďte kontrolu výpočtu dosazením do sestavené diferenciální rovnice. 

\begin{table}[ht]
  \begin{center}
    \begin{tabular}{|c|c|c|c|c|c|c|c|c|c|} 
      \hline
      %prvni radek
       sk. & $U$ [\si{\volt}] & $L$ [\si{H}] & $R$ [\si{\ohm}] & $i_L(0)$ [\si{\ampere}]\\
       %druhy radek
       \hline
       E & 40 & 30 & 40 & 11\\
     \hline
    \end{tabular}
    \caption{Zadané hodnoty}
    \label{tab:5}
  \end{center}
 \end{table}
 
%obvod 36
\begin{figure}[ht!]
\begin{center}
\begin{circuitikz}
    \draw
    (0,0) to[dcvsource, v^<=$U$] ++(0,2.5)
    to[cosw, l=$S$] (0,5)
    to[R=$R$] (4,5)
    to[cute inductor, l=$L$, i=$i_L$] (4,0) -- (0,0);
     
    \draw
    node (A) at (-0.6,4.1) {}
    node (B) at (0.07,4.1) {}
     [thin,->,=triangle 45] (A) to[bend left] (B);
   
    \draw
    node (A) at (-0.6,4.5) {\small $t = 0s$};
    
\end{circuitikz}
\caption{Zadaný obvod}
\end{center}
\end{figure}

\subsection{Sestavení diferenciální rovnice prvního řádu}
Nejprve si vyjádříme vztahy platicí v zadaném obvodu:
\[i_L = \frac{u_R}{R} \hspace{1cm} \left(i = i_L = i_R\right)\]
\[u_R + u_L - U = 0\]
\[i'_L = \frac{u_L}{L}\]
\newline
Dále si zavedeme počáteční podmínku:
\[i_L(0) = 11 \si{\ampere}\]
\newline
Nyní využijeme vyjádřené vztahy a sestavíme si diferenciální rovnici prvního řádu pro $i'_L$:
\[Ri_L +  Li'_L = U\]
\[i'_L = \frac{1}{L} * \left( U - Ri_L \right)\] 

\newpage
\subsection{Analytické řešení}
Očekávané řešení:
\[i_L(t) = K(t) * e^{\lambda t}\]
\newline 
Řešíme charakteristické rovnice ($i'_L = \lambda$, $i_L = 1$):
\[R + L\lambda = 0\]
\[\lambda = -\frac{R}{L} = -\frac{4}{3}\]
\newline
Dosadíme $\lambda$ do očekávaného řešení:
\[i_L(t) = K(t) * e^{\lambda t}\]
\[i_L(t) = K(t) * e^{-\frac{R}{L}t}\]
\newline
Dále provedeme derivaci získané rovnice:
\[i'_L = K'(t) * e^{-\frac{R}{L}t} + K(t) * \left( -\frac{R}{L} \right) *  e^{-\frac{R}{L}t}\]
\newline
Nyní dosadíme získané rovnice do námi sestavené diferenciální rovnice:
\[Ri_L +  Li'_L = U\]
\[R *  K(t) * e^{-\frac{R}{L}t} +  L * \left(  K'(t) * e^{-\frac{R}{L}t} + K(t) * \left( -\frac{R}{L} \right) *  e^{-\frac{R}{L}t} \right) = U\]
\[R *  K(t) * e^{-\frac{R}{L}t} +  L * K'(t) * e^{-\frac{R}{L}t} + L * K(t) * \left( -\frac{R}{L} \right) *  e^{-\frac{R}{L}t} = U\]
\[R *  K(t) * e^{-\frac{R}{L}t} +  L * K'(t) * e^{-\frac{R}{L}t} - R * K(t) * e^{-\frac{R}{L}t} = U\]
\[L * K'(t) * e^{-\frac{R}{L}t} = U\]
\[K'(t) * e^{-\frac{R}{L}t} = \frac{U}{L}\]
\[K'(t) = \frac{U}{L} * e^{\frac{R}{L}t}\]
\newline
Nyní známe $K'(t)$. Potřebujeme však zjistit $K(t)$, proto rovnici zintegrujeme:
\[K(t) =\int \frac{U}{L} * e^{\frac{R}{L}t} dt\]
\[K(t) = \frac{U * e^{\frac{R}{L}t}}{R} + k\]
\newline
Dosadíme $K(t)$ do očekávaného řešení:
\[i_L(t) = \left( \frac{U * e^{\frac{R}{L}t}}{R} + k \right) * e^{\lambda t}\]
\begin{equation}
    \label{eq:rovnice}
    i_L(t) = \frac{U}{R} + k * e^{-\frac{R}{L}t}
\end{equation}
\newpage
\noindent
Dále dosadíme počáteční podmínku $i_L(0) = 11\si{\ampere}$:
\[11 = \frac{U}{R} + k * e^{-\frac{R}{L}0}\]
\[11 = \frac{U}{R} + k\]
\[k = 11 - \frac{U}{R}\]
Dosadíme k do rovnice (\ref{eq:rovnice}):
\[i_L(t) = \frac{U}{R} + k * e^{-\frac{R}{L}t}\]
\[i_L(t) = \frac{U}{R} + \left(11 - \frac{U}{R}\right) * e^{-\frac{R}{L}t}\]
\newline
Nyní dosadíme číselné hodnoty:
\[i_L(t) = \frac{40}{40} + \left(11 - \frac{40}{40}\right) * e^{-\frac{40}{30}t}\]
\newline
Hledaná rovnice tedy je:
\[i_L(t) = 1 + 10 * e^{-\frac{4}{3}t}\]

\vspace{1cm}
\subsection{Zkouška}
\begin{enumerate}[label=\alph*)]
    \item 
    \[t = 0s: \hspace{1cm} i_L(0) = \frac{U}{R} + 11 - \frac{U}{R} = 11\]
    
    \item
    Dosadíme $i_L$ a $i'_L$ do diferenciální rovnice prvního řádu a upravíme:
    \[Ri_L +  Li'_L = U\]
    \[i_L(t) = \frac{U}{R} + \left(11 - \frac{U}{R}\right) * e^{-\frac{R}{L}t}\]
    \[i'_L(t) = -\left(11 - \frac{U}{R}\right) * \frac{R}{L} * e^{-\frac{R}{L}t}\]
    \vspace{1cm}
    \[R * \left[ \frac{U}{R} + \left(11 - \frac{U}{R}\right) * e^{-\frac{R}{L}t} \right] +  L * \left[ -\left(11 - \frac{U}{R}\right) * \frac{R}{L} * e^{-\frac{R}{L}t} \right] = U\]
    \[U + R * \left(11 - \frac{U}{R}\right) * e^{-\frac{R}{L}t} - R * \left(11 - \frac{U}{R}\right) * e^{-\frac{R}{L}t} = U\]
    \[U = U\]
    \[0 = 0\]
\end{enumerate}

\newpage
%Priklad 5
\rhead{\bfseries Výsledky}

\section{Výsledky}
\begin{table}[htbp]
  \begin{center}
    \begin{tabular}{|c|c|c|} 
      \hline
      %prvni radek
       Příklad & Varianta zadání & Výsledek\\
       \hline
       
       \multirow[t]{2}{*}{1} & H & $I_{R_6} = \SI{0,0821}{\ampere}$\\
        & & $U_{R_6} = \SI{71,4187}{\volt}$\\ 
        
       %{$$I_{R_6} = \SI{0,0821}{\ampere}$$ \\ $$U_{R_6} = \SI{71,4187}{\volt}$$}\\
       \hline
       \multirow[t]{2}{*}{2} & E & $I_{R_3} = \SI{0,1577}{\ampere}$\\
        & & $U_{R_3} = \SI{98,5636}{\volt}$ \\ 
       \hline
       \multirow[t]{2}{*}{3} & D & $I_{R_2} = \SI{1,3087}{\ampere}$\\
        & & $U_{R_2} = \SI{49,729 8}{\volt}$ \\ 
       \hline
       \multirow[t]{2}{*}{4} & H & $|U_{L_2}| = \SI{22,0663}{\volt}$\\
        & & $\varphi_{L_2} = \SI{-0,3964}{rad}$ \\ 
       \hline
       \multirow[t]{2}{*}{5} & E & $i_L(t) = 1 + 10 * e^{-\frac{4}{3}t}$\\
        & & \\ 
     \hline
    \end{tabular}
    \caption{Tabulka s výsledky}
    \label{tab:6}
  \end{center}
 \end{table}
\end{document}